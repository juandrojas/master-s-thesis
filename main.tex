\documentclass[12pt, partial, oneside]{aucklandthesis}
%
% This is a template for University of Auckland theses.
%
% Written by Alistair Kwan, June 2016
% 
%
% Options:
%	10pt, 11pt, 12pt: size of main text
% 	examcopy: asserts confidentiality for examination copies
%	partial: thesis partial fulfils degree requirements
%	singlespace, onehalfspace, doublespace: line spacing
%	oneside: format for single-sided printing
%	draft: adds 'draft' and date to footer
%

%
% Add, delete or un-comment packages below as required.
%

\usepackage[utf8]{inputenc}
\usepackage[T1]{fontenc}
\usepackage{todonotes}
%\usepackage{fullpage}

%\usepackage{graphicx} % for inserting graphics files
\usepackage{appendix} % for appendices

%\usepackage{hyperref} % for formatting web addresses and other URLs
%\urlstyle{same} % try also tt, sf if this option doesn't produce clear enough output

% Readability options
%
\usepackage{booktabs} % for table rules
%\usepackage{microtype} % for improved justification

% Typeface options — choose one if desired
% or choose a different typeface to accommmodate character sets
% as needed for East Asian and other languages.
%
% Consider compiling using the XeLaTeX engine if you have more extreme
% typeface needs, e.g. for multiple languages, or a need for symbols particular
% to a typeface.
%
% See also the LaTeX Symbols List at
% https://http://www.ctan.org/pkg/comprehensive
%
%\usepackage{mathptmx} % Times New Roman, including mathematics
%\usepackage{mathpazo} % Palatino with mathematics support
%\usepackage{fourier} % Utopia, a serif typeface with Fourier mathematics
%\usepackage{gentium} % a contemporary serif typeface
%\usepackage{libertine} % a softer-feeling serif typeface; also installs sans-serif font Biolinum
%\usepackage{fouriernc} % Century Schoolbook with Fourier maths
%\usepackage{mathpple} % Palatino with Fourier maths
\usepackage{baskervillef}


% To set the sans serif font (for \sffamily):
%\usepackage[scaled]{helvet} % Nimbus, like Helvetica
%\usepackage{universalis} % Universalis
%\usepackage{avant} % URW Gothic, like Avant Garde
%\usepackage{PTSansNarrow}
%\usepackage{AlegreyaSans} % Alegreya Sans

% To set the mathematics font:
%\usepackage{eulervm} % Euler, based on a Zapf design

% To set the (usually monospaced) typewriter font:
%\usepackage[ttdefault=true]{AnonymousPro}
%\usepackage[scaled]{beramono}
%\usepackage{inconsolata}
%\usepackage{sourcecodepro}

%Standard math packages

\usepackage{amsmath} 
\usepackage{amsthm} 
\usepackage{amssymb}
\usepackage{amsfonts}
\usepackage{graphicx}
\usepackage{mathrsfs} % activates \mathscr
\usepackage{tikz-cd}
\usepackage{enumitem} % activates custom enumeration
\usepackage{mathtools}
\usepackage{eucal}
\usetikzlibrary{arrows, matrix}

\chapterstyle{Forder}

%\usepackage{cjk} % for Chinese, Japanese, Korean

%\usepackage{tabularx} % For easier table formatting.

%\usepackage[nottoc]{tocbibind} % Controls the table of contents
%   nottoc: don't list table of contents inside itself
%   section: go as far as section-level headings

% Automated bibliography
%
\usepackage[maxbibnames=9,maxcitenames=3,maxalphanames=3,style=alphabetic,backref]{biblatex}
\addbibresource{referencias.bib}
\usepackage[hyperfootnotes=false]{hyperref}
\usepackage[capitalize,noabbrev,nameinlink]{cleveref}
%bibliography{bibliography1.bib, bibliography2.bib} % Specify bibliography files 

% Enviroments 

\numberwithin{equation}{chapter}

\theoremstyle{definition}
\newtheorem{definition}[equation]{Definition}
\newtheorem{example}[equation]{Example}
\newtheorem{notation}[equation]{Notation}

\theoremstyle{plain}
\newtheorem{theorem}[equation]{Theorem}
\newtheorem{corollary}[equation]{Corollary}
\newtheorem{lemma}[equation]{Lemma}
\newtheorem{proposition}[equation]{Proposition}

\theoremstyle{remark}
\newtheorem{remark}[equation]{Remark}


\newcommand{\id}[1]{\operatorname{id}_{#1}}

%% MATH OPS

\def\N{{\mathbb N}}
\def\Z{{\mathbb Z}}
\def\F{{\mathbb F}}
\def\Q{{\mathbb Q}}
\def\R{{\mathbb R}}
\def\C{{\mathbb C}}
\def\O{{\mathscr O}}
\def\A{{\mathbb A}}

\newcommand{\hbarr}[0]{[[\hbar]]}

%% auxiliary

\DeclareMathOperator{\Der}{Der}

\DeclareMathOperator*{\colim}{co{\lim}}
\makeatletter
\patchcmd{\varlim@}{lim}{\lim}{}{}
\makeatother

\begin{document}

% ====================================================
%
% FRONTMATTER
%
% Arabic pagination, starting with the title page
% which is counted but not numbered
%
% ====================================================

% Specify the title page content
\title{Canonical quantization of symplectic manifolds}
%\subtitle{[subtitle]}
\author{Juan Diego Rojas}
\degreesought{Master of Arts} 
\degreediscipline{Mathematics}
\degreecompletionyear{2020}

% Print the title page
\maketitle

% Dedication (optional)
%\thesisdedication{Dedicada a mamá y papá.}

% Preface and/or acknowledgements (optional)
%\input{acknowledgements} % it's in a separate file

% Contents, lists of tables and figures
\settocdepth{subsection} % choose chapter, section, subsection \cleardoublepage\tableofcontents
%\cleardoublepage\listoffigures
%\cleardoublepage\listoftables

% Glossary (optional)
%\input{glossary}

% ====================================================
%
% MAINMATTER
%
% Include external chapter files here using
% the \input{} command
%
% If you run out of memory during compilation,
% switch some or all chapters to \include{} instead of \input{}, 
% but watch out for pagination problems.
%
% ====================================================

%%!TEX root = main.tex
\chapter*{Abstract}\label{ch:abstract}
Cuantización por deformación para una variedad de Poisson se define en términos de un producto estrella; es decir, es una deformación uniparamétrica del haz de (\textit{sheaf of}) álgebras de funciones suaves. Recordamos la idea esencial. Sea $(X, \mathscr{O}_{X})$ una variedad simpléctica, un producto asociativo $\star$ en $\mathscr{O}_X[[\hbar]]$, dónde $\hbar$ es un indeterminado, es un \textit{producto estrella} si es $\mathbb{C}[[\hbar]]$-bilineal y satisface
\[
        f \star g = \sum_{i\geq 0}P_i(f,g)\hbar^i\, \text{ para }f,g\in\mathscr{O}_X,
        \]
       	dónde $P_i$ son operadores bi-diferenciales tales que $P_0(f,g) = fg$ y $P_i(f,1) = P_i(1,f) = 0$. Un producto estrella define una estructura de Poisson mediante la regla
        \[
        \{f,g\} := P_1(f,g) - P_1(g,f) = \frac{f\star g - g\star f}{\hbar} \text{ mod } \hbar\mathscr{O}_X.
        \]
Si $X$ es Poisson y $\star$ induce el corchete de Poisson correspondiente en $X$, decimos que $\star$ \textit{deforma} la estructura de Poisson de $X$. B. V. Fedosov (1985) y De Wilde-Lecomte (1983, 1985) demostraron, de manera independiente, que el caso simpléctico admite tal deformación de su corchete de Poisson simpléctico.

A diferencia de variedades de Poisson generales, para variedades simplécticas el teorema de Darboux asegura una estructura local simple: localmente toda variedad simpléctica admite un sistema de coordenadas (llamadas coordenadas de Darboux) $(q_i, p_i)$ en las cuales la forma simpléctica toma la forma
\[
    \omega = \sum dq_i \wedge dp_i.
\]
Estas coordenadas admiten una deformación conocida como el producto de \textit{Moyal-Weyl}  \[
    f \star g=\operatorname{prod}(\exp (\hbar \pi)(f \otimes g))=f \cdot g+\hbar\{f, g\}+\cdots
    \]
dónde $\pi$ es el bi-vector de Poisson.    

En \cite{deligne-deformations} P. Deligne comparó las construcciones de B. V. Fedosov y De Wilde-Lecomte y produjo resultados sobre la equivalencia de ambos productos estrelllas en el lenguaje de cohomología no abeliana usando cociclos de Čech. La idea esencial consiste en introducir estructuras asociadas a un producto estrella
\begin{enumerate}[label = (\roman*)]
	\item Observe que la inclusión $\mathbb{C}[[\hbar]] \to \mathscr{O}_X[[\hbar]]$ es inyectiva y sobreyecta al centro, de tal manera que $\mathbb{C}[[\hbar]] \simeq \operatorname{Z}(\mathscr{O}_X[[\hbar]])$. Estudiamos el problema de encontrar $\mathbb{C}[[t]]$-derivaciones de $\mathscr{O}_X[[\hbar]]$ que se restringen al campo de Euler $\hbar\partial_\hbar$. Esto es precisamente preguntar cuándo la dilatación $(p,q,\hbar) \to (\lambda^{1/2}p, \lambda^{1/2}q, \lambda \hbar)$ forma un subgrupo uniparamétrico de automorfismos de $(\mathscr{O}_X[[t]],\star)$. A esta propiedad la llamamos \textit{equivarianza por dilatación};
	\item La involución $\hbar \to -\hbar$ induce un producto opuesto $\star_{-\hbar}$. Decimos que un producto estrella $\star$ es \textit{invariante bajo transposición} si $f \star_{-\hbar} g = g \star_{\hbar} f$,
\end{enumerate}
y notar que el producto de Moyal-Weyl satisface ambas propiedades y es caracterizado por ellas. Este hecho junto con el teorema de Darboux reduce el problema de clasificación de productos estrellas a un problema de cohomología con coeficientes no abelianos. En adición, de los trabajos de M. Kontsevich \cite{kontsevich} y M. Kashiwara \cite{Kashiwara-contact} se volvió evidente que el contexto más general de \textit{stacks} de algebroides permite un desarrollo más natural de la teoría. De esta forma se introdujo a la noción de \textit{DQ-algebroide} debida a M. Kashiwara y P. Schapira \cite{DQM}. 

El objetivo de este trabajo es producir una clasificación de DQ-algebroides simplécticos mediante la introducción de las estructuras previamente mencionadas y el lenguaje de \cite{quasi-classical}.
%\addcontentsline{toc}{chapter}{abstract}


%!TEX root = ../../main.tex
\chapter{dq-algebras and dq-algebroids}\label{ch:DQ}
\section{Star products}
\todo{some words on history...}
\begin{definition}[{\cite{BFFLS}, \cite[Definition~2.2.2.]{DQM}}]\label{def:star-product}
	An associative operation $\star$ on $\mathscr{O}_{X}[[\hbar]]$ is a \textit{star product} if it is $\C[[\hbar]]$-bilinear and satisfies
	\begin{align}\label{eqn:star-product}
		f \star g = \sum_{i\geq 0} P_{i}(f,g)\hbar^{i} \quad \text{for } f,g\in \mathscr{O}_{X}
	\end{align}
	where the $P_{i}$'s are bi-differential operators such that $P_{0}(f, g)=fg$ and $P_{i}(f, 1)=P_{i}(1, f)=0$ for all $f \in \mathscr{O}_{X}$ and $i>0 .$ We call $\left(\mathscr{O}_{X}[[\hbar]], \star\right)$ a \textit{star algebra}.
\end{definition}
\todo{the previous probably needs some background on differential operators...}
\subsection{Moyal-Weyl star product}
\begin{definition}\label{def:weyl-moyal}
	Let $X = \mathbb{R}^{2n}$ endowed with its standard symplectic structure. We define the \textit{Moyal-Weyl} $\star$-\textit{product} by the rule
    \begin{align}\label{eqn:weyl-moyal-free}
    	f \star g=\operatorname{prod}\left(\exp \left(\frac{i\hbar}{2}\Pi\right)(f \otimes g)\right)
    \end{align}
	where $\Pi$ is the Poisson bi-vector. One calls $\mathscr{O}_X[[\hbar]]$ equipped with the Moyal-Weyl star product, the \textit{Weyl algebra}.
\end{definition}
\begin{remark}\label{rem:weyl-algebra}
	The notation Weyl algebra in \cref{def:weyl-moyal} is supported in the following observation. Consider the subalgebra $(\C[p,q], \star)$\footnote{Here $p$ and $q$ are shorthand for coordinates $(p_{1}, \ldots, p_{n}, q_{1}, \ldots, q_{n})$} of the algebra $(\mathscr{O}_{\R^{2n}}(\R^{2n}), \star)$. An easy computation yields 
	\[
		p_{r} \star q_{s} = p_{s}q_{s} + \delta_{rs}\frac{i\hbar}{2} \quad \text{and,} \quad q_{s} \star p_{r} = p_{r}q_{s} - \delta_{sr}\frac{i\hbar}{2}.
	\]
	Therefore,
	\[
		[p_{r},q_{s}] = \delta_{rs}i\hbar,
	\]
	which is \textit{Planck’s law}. Particularly, this presents an isomorphism between $(\C[p,q],\star)$ and\footnote{We mantain the same shorthand as before, and we extend it correctly to $x\partial - \partial x - 1$} $\C\{x,\partial\}/\left\langle x\partial - \partial x - 1\right\rangle$, also known as the Weyl algebra. This will be further explained by the identification of $\R^{2n}$ with $T^{*}\R^{n}$ and deformation quantization of the cotangent bundle. In the meantime, this identification is proven useful in the following Lemma. \todo{write reference when cotangent bundle is explained.}
\end{remark}
\begin{lemma}\label{lemm:center-weyl-algebra}
	The center of the algebra $\C\{x,\partial\}/\left\langle x\partial - \partial x - 1\right\rangle$ is $\C$. In particular, from \cref{rem:weyl-algebra}, it follows that the center of $(\mathscr{O}_{\R^{2n}}, \star)$ is $\C[[t]]$.
\end{lemma}
\begin{proof}
	We carry out the case $n=1$. Let $P = \sum_{a}f_{a}(x)\partial^{a}$ be a central element. An easy calculation, using the fact that $[\partial, f] = f'$, yields
	\[
		[\partial, P] = \sum_{a}f_{a}'(x)\partial^{a}.
	\]
	Since $P$ is central, we conclude that $f_{a}' = 0$, so that $f_{a}$ is a constant for all $a$. Therefore, $P = g(\partial)$ for some polynomial $g$. Again, we have
	\[
		[x, g(\partial)] = g'(\partial)
	\]
	and centrality yields that $P$ is constant. The previous proof only uses that the ground ring of the Weyl algebra is of characteristic zero, so that one carries inductively using $\C[x, \partial_{x}][y_{1}, \ldots, y_{m}, \partial_{1}, \ldots, \partial_{m}] \cong \C[x, y_{1}, \ldots, y_{m}, \partial_{x}, \partial_{1}, \ldots, \partial_{m}].$
\end{proof}

\section{DQ-algebras}
See Kashiwara-Schapira.
\section{DQ-algebroids}
See Kashiwara-Schapira. Important here to include some motivation. 
%!TEX root = ../../main.tex
\chapter{Dilation equivariance}\label{ch:dilation}
Let $(X, \omega)$ be a symplectic manifold and $\mathscr{A}$ a symplectic DQ-algebra on $X$. Denote by $Z(\mathscr{A})$ the center of $\mathscr{A}$. From \cref{lemm:center-weyl-algebra} and \todo{include reference of DQ-Darboux} the natural map $\C[ [\hbar]] \to \mathscr{A}$ is injective onto the center.


\appendix
%!TEX root = ../../main.tex
\chapter{Abstract nonsense}\label{app:abstract-nonsense}
Once and for all fix a strongly inaccesible cardinal $\kappa$, so that $V_{\kappa}$ is a Grothendieck universe. (hahaha)

% ====================================================
%
% ENDMATTER
%
% Appendices and bibliography 
% Pagination arabic, re-starts at 1
%
% ====================================================
\cleardoublepage % start afresh on a new page
%\appendixpage* % makes a page to mark beginning of appendices
% \input{appendix1} 


\printbibliography[title={References}, heading=bibintoc]


\end{document}