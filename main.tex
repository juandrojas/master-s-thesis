\documentclass[12pt, partial, oneside]{aucklandthesis}
%
% This is a template for University of Auckland theses.
%
% Written by Alistair Kwan, June 2016
% 
%
% Options:
%	10pt, 11pt, 12pt: size of main text
% 	examcopy: asserts confidentiality for examination copies
%	partial: thesis partial fulfils degree requirements
%	singlespace, onehalfspace, doublespace: line spacing
%	oneside: format for single-sided printing
%	draft: adds 'draft' and date to footer
%

%
% Add, delete or un-comment packages below as required.
%

\usepackage[utf8]{inputenc}
\usepackage[T1]{fontenc}
\usepackage{todonotes}
%\usepackage{fullpage}

%\usepackage{graphicx} % for inserting graphics files
\usepackage{appendix} % for appendices

%\usepackage{hyperref} % for formatting web addresses and other URLs
%\urlstyle{same} % try also tt, sf if this option doesn't produce clear enough output

% Readability options
%
\usepackage{booktabs} % for table rules
%\usepackage{microtype} % for improved justification

% Typeface options — choose one if desired
% or choose a different typeface to accommmodate character sets
% as needed for East Asian and other languages.
%
% Consider compiling using the XeLaTeX engine if you have more extreme
% typeface needs, e.g. for multiple languages, or a need for symbols particular
% to a typeface.
%
% See also the LaTeX Symbols List at
% https://http://www.ctan.org/pkg/comprehensive
%
%\usepackage{mathptmx} % Times New Roman, including mathematics
%\usepackage{mathpazo} % Palatino with mathematics support
%\usepackage{fourier} % Utopia, a serif typeface with Fourier mathematics
%\usepackage{gentium} % a contemporary serif typeface
%\usepackage{libertine} % a softer-feeling serif typeface; also installs sans-serif font Biolinum
%\usepackage{fouriernc} % Century Schoolbook with Fourier maths
%\usepackage{mathpple} % Palatino with Fourier maths
\usepackage{baskervillef}


% To set the sans serif font (for \sffamily):
%\usepackage[scaled]{helvet} % Nimbus, like Helvetica
%\usepackage{universalis} % Universalis
%\usepackage{avant} % URW Gothic, like Avant Garde
%\usepackage{PTSansNarrow}
%\usepackage{AlegreyaSans} % Alegreya Sans

% To set the mathematics font:
%\usepackage{eulervm} % Euler, based on a Zapf design

% To set the (usually monospaced) typewriter font:
%\usepackage[ttdefault=true]{AnonymousPro}
%\usepackage[scaled]{beramono}
%\usepackage{inconsolata}
%\usepackage{sourcecodepro}

%Standard math packages

\usepackage{amsmath} 
\usepackage{amsthm} 
\usepackage{amssymb}
\usepackage{amsfonts}
\usepackage{graphicx}
\usepackage{mathrsfs} % activates \mathscr
\usepackage{tikz-cd}
\usepackage{enumitem} % activates custom enumeration
\usepackage{mathtools}
\usepackage{eucal}
\usetikzlibrary{arrows, matrix}

\chapterstyle{Forder}

%\usepackage{cjk} % for Chinese, Japanese, Korean

%\usepackage{tabularx} % For easier table formatting.

%\usepackage[nottoc]{tocbibind} % Controls the table of contents
%   nottoc: don't list table of contents inside itself
%   section: go as far as section-level headings

% Automated bibliography
%
\usepackage[maxbibnames=9,maxcitenames=3,maxalphanames=3,style=alphabetic,backref]{biblatex}
\addbibresource{referencias.bib}
\usepackage[hyperfootnotes=false]{hyperref}
\usepackage[capitalize,noabbrev,nameinlink]{cleveref}
%bibliography{bibliography1.bib, bibliography2.bib} % Specify bibliography files 

% Enviroments 

\numberwithin{equation}{chapter}

\theoremstyle{definition}
\newtheorem{definition}[equation]{Definition}
\newtheorem{example}[equation]{Example}
\newtheorem{notation}[equation]{Notation}

\theoremstyle{plain}
\newtheorem{theorem}[equation]{Theorem}
\newtheorem{corollary}[equation]{Corollary}
\newtheorem{lemma}[equation]{Lemma}
\newtheorem{proposition}[equation]{Proposition}

\theoremstyle{remark}
\newtheorem{remark}[equation]{Remark}


\newcommand{\id}[1]{\operatorname{id}_{#1}}

%% MATH OPS

\def\N{{\mathbb N}}
\def\Z{{\mathbb Z}}
\def\F{{\mathbb F}}
\def\Q{{\mathbb Q}}
\def\R{{\mathbb R}}
\def\C{{\mathbb C}}
\def\O{{\mathscr O}}
\def\A{{\mathbb A}}

\newcommand{\hbarr}[0]{[[\hbar]]}

%% auxiliary

\DeclareMathOperator{\Der}{Der}

\DeclareMathOperator*{\colim}{co{\lim}}
\makeatletter
\patchcmd{\varlim@}{lim}{\lim}{}{}
\makeatother

\begin{document}

% ====================================================
%
% FRONTMATTER
%
% Arabic pagination, starting with the title page
% which is counted but not numbered
%
% ====================================================

% Specify the title page content
\title{Canonical quantization of symplectic manifolds}
%\subtitle{[subtitle]}
\author{Juan Diego Rojas}
\degreesought{Master of Arts} 
\degreediscipline{Mathematics}
\degreecompletionyear{2020}

% Print the title page
\maketitle

% Dedication (optional)
%\thesisdedication{Dedicada a mamá y papá.}

% Preface and/or acknowledgements (optional)
%\input{acknowledgements} % it's in a separate file

% Contents, lists of tables and figures
\settocdepth{subsection} % choose chapter, section, subsection \cleardoublepage\tableofcontents
%\cleardoublepage\listoffigures
%\cleardoublepage\listoftables

% Glossary (optional)
%\input{glossary}

% ====================================================
%
% MAINMATTER
%
% Include external chapter files here using
% the \input{} command
%
% If you run out of memory during compilation,
% switch some or all chapters to \include{} instead of \input{}, 
% but watch out for pagination problems.
%
% ====================================================

%%!TEX root = main.tex
\chapter*{Abstract}\label{ch:abstract}
Cuantización por deformación para una variedad de Poisson se define en términos de un producto estrella; es decir, es una deformación uniparamétrica del haz de (\textit{sheaf of}) álgebras de funciones suaves. Recordamos la idea esencial. Sea $(X, \mathscr{O}_{X})$ una variedad simpléctica, un producto asociativo $\star$ en $\mathscr{O}_X[[\hbar]]$, dónde $\hbar$ es un indeterminado, es un \textit{producto estrella} si es $\mathbb{C}[[\hbar]]$-bilineal y satisface
\[
        f \star g = \sum_{i\geq 0}P_i(f,g)\hbar^i\, \text{ para }f,g\in\mathscr{O}_X,
        \]
       	dónde $P_i$ son operadores bi-diferenciales tales que $P_0(f,g) = fg$ y $P_i(f,1) = P_i(1,f) = 0$. Un producto estrella define una estructura de Poisson mediante la regla
        \[
        \{f,g\} := P_1(f,g) - P_1(g,f) = \frac{f\star g - g\star f}{\hbar} \text{ mod } \hbar\mathscr{O}_X.
        \]
Si $X$ es Poisson y $\star$ induce el corchete de Poisson correspondiente en $X$, decimos que $\star$ \textit{deforma} la estructura de Poisson de $X$. B. V. Fedosov (1985) y De Wilde-Lecomte (1983, 1985) demostraron, de manera independiente, que el caso simpléctico admite tal deformación de su corchete de Poisson simpléctico.

A diferencia de variedades de Poisson generales, para variedades simplécticas el teorema de Darboux asegura una estructura local simple: localmente toda variedad simpléctica admite un sistema de coordenadas (llamadas coordenadas de Darboux) $(q_i, p_i)$ en las cuales la forma simpléctica toma la forma
\[
    \omega = \sum dq_i \wedge dp_i.
\]
Estas coordenadas admiten una deformación conocida como el producto de \textit{Moyal-Weyl}  \[
    f \star g=\operatorname{prod}(\exp (\hbar \pi)(f \otimes g))=f \cdot g+\hbar\{f, g\}+\cdots
    \]
dónde $\pi$ es el bi-vector de Poisson.    

En \cite{deligne-deformations} P. Deligne comparó las construcciones de B. V. Fedosov y De Wilde-Lecomte y produjo resultados sobre la equivalencia de ambos productos estrelllas en el lenguaje de cohomología no abeliana usando cociclos de Čech. La idea esencial consiste en introducir estructuras asociadas a un producto estrella
\begin{enumerate}[label = (\roman*)]
	\item Observe que la inclusión $\mathbb{C}[[\hbar]] \to \mathscr{O}_X[[\hbar]]$ es inyectiva y sobreyecta al centro, de tal manera que $\mathbb{C}[[\hbar]] \simeq \operatorname{Z}(\mathscr{O}_X[[\hbar]])$. Estudiamos el problema de encontrar $\mathbb{C}[[t]]$-derivaciones de $\mathscr{O}_X[[\hbar]]$ que se restringen al campo de Euler $\hbar\partial_\hbar$. Esto es precisamente preguntar cuándo la dilatación $(p,q,\hbar) \to (\lambda^{1/2}p, \lambda^{1/2}q, \lambda \hbar)$ forma un subgrupo uniparamétrico de automorfismos de $(\mathscr{O}_X[[t]],\star)$. A esta propiedad la llamamos \textit{equivarianza por dilatación};
	\item La involución $\hbar \to -\hbar$ induce un producto opuesto $\star_{-\hbar}$. Decimos que un producto estrella $\star$ es \textit{invariante bajo transposición} si $f \star_{-\hbar} g = g \star_{\hbar} f$,
\end{enumerate}
y notar que el producto de Moyal-Weyl satisface ambas propiedades y es caracterizado por ellas. Este hecho junto con el teorema de Darboux reduce el problema de clasificación de productos estrellas a un problema de cohomología con coeficientes no abelianos. En adición, de los trabajos de M. Kontsevich \cite{kontsevich} y M. Kashiwara \cite{Kashiwara-contact} se volvió evidente que el contexto más general de \textit{stacks} de algebroides permite un desarrollo más natural de la teoría. De esta forma se introdujo a la noción de \textit{DQ-algebroide} debida a M. Kashiwara y P. Schapira \cite{DQM}. 

El objetivo de este trabajo es producir una clasificación de DQ-algebroides simplécticos mediante la introducción de las estructuras previamente mencionadas y el lenguaje de \cite{quasi-classical}.
%\addcontentsline{toc}{chapter}{abstract}


%!TEX root = ../../main.tex
\chapter{dq-algebras and dq-algebroids}\label{ch:DQ}
\section{Star products}
\todo{some words on history...}
\begin{definition}[{\cite{BFFLS}, \cite[Definition~2.2.2.]{DQM}}]\label{def:star-product}
	An associative operation $\star$ on $\mathscr{O}_{X}[[\hbar]]$ is a \textit{star product} if it is $\C[[\hbar]]$-bilinear and satisfies
	\begin{align}\label{eqn:star-product}
		f \star g = \sum_{i\geq 0} P_{i}(f,g)\hbar^{i} \quad \text{for } f,g\in \mathscr{O}_{X}
	\end{align}
	where the $P_{i}$'s are bi-differential operators such that $P_{0}(f, g)=fg$ and $P_{i}(f, 1)=P_{i}(1, f)=0$ for all $f \in \mathscr{O}_{X}$ and $i>0 .$ We call $\left(\mathscr{O}_{X}[[\hbar]], \star\right)$ a \textit{star algebra}. A star-product defines a Poisson structure on $X$ by the rule
	\begin{align}\label{eqn:poisson-induced-by-star-product}
		\{f,g\} := P_{1}(f,g) - P_{1}(f,g) = \frac{1}{\hbar}[f,g]_{\star} \mod \hbar\mathscr{O}_{X} 
	\end{align}
	If $X$ is a Poisson manifold and the induced Poisson bracket by $\star$ coincides with the Poisson structure of $X$, we say that $\star$ is a \textit{deformation} of the Poisson structure (bracket) on $X$. Moreover, if $(X,\omega)$ is symplectic and $\star$ deforms the standard Poisson bracket induced by $\omega$, we say that $\star$ is a \textit{symplectic star product}.
\end{definition}
\todo{the previous probably needs some background on differential operators...}
\subsection{Moyal product}
\begin{definition}\label{def:moyal}
	Let $X = \mathbb{R}^{2n}$ endowed with its standard symplectic structure. We define the \textit{Moyal product} by the rule
    \begin{align}\label{eqn:weyl-moyal-free}
    	f \star g &= \operatorname{prod}\left(\exp \left(\frac{\hbar}{2}\Pi\right)(f \otimes g)\right) \\
    	&= fg + \frac{\hbar}{2}\sum_{i,j}\Pi^{ij}(\partial_{i}f)(\partial_{j}g) + \frac{\hbar^{2}}{8} \sum_{i,j,k,m} \Pi^{ij}\Pi^{km}(\partial_{i}\partial_{k}g)(\partial_{j}\partial_{m}g) + \ldots, \nonumber
    \end{align}
	where $\Pi$ is the Poisson bi-vector. One calls $\mathscr{O}_X[[\hbar]]$ equipped with the Moyal product, the \textit{Weyl algebra}.
\end{definition}
\begin{remark}\label{rem:weyl-algebra}
	The notation Weyl algebra in \cref{def:moyal} is supported in the following observation. Consider the subalgebra $(\C[p,q], \star)$\footnote{Here $p$ and $q$ are shorthand for coordinates $(p_{1}, \ldots, p_{n}, q_{1}, \ldots, q_{n})$} of the algebra $(\mathscr{O}_{\R^{2n}}(\R^{2n}), \star)$. An easy computation yields 
	\[
		p_{r} \star q_{s} = p_{s}q_{s} + \delta_{rs}\frac{\hbar}{2} \quad \text{and,} \quad q_{s} \star p_{r} = p_{r}q_{s} - \delta_{sr}\frac{\hbar}{2}.
	\]
	Therefore,
	\[
		[p_{r},q_{s}]_{\star} = \delta_{rs}\hbar.
	\]
	Particularly, this presents an isomorphism between $(\C[p,q],\star)$ and\footnote{We mantain the same shorthand as before, and we extend it correctly to $x\partial - \partial x - 1$} $\C\{x,\partial\}/\left\langle x\partial - \partial x - 1\right\rangle$, also known as the Weyl algebra. This will be further explained by the identification of $\R^{2n}$ with $T^{*}\R^{n}$ and deformation quantization of the cotangent bundle.
\end{remark}
\begin{lemma}\label{lemm:center-weyl-algebra}
	The center of the algebra $(\mathscr{O}_{\R^{2n}}, \star)$ is $\C[[\hbar]]$. 
\end{lemma}
\begin{proof}
	Choose coordinates $(p_{i},q_{i})$ and let $f \in (\mathscr{O}_{\R^{2n}}, \star)$ be central. Then
	\[
		0 = [f,q_{i}]_{\star} = -\hbar\frac{\partial f}{\partial q_{i}}\quad \text{and,}\quad \hbar\frac{\partial f}{\partial q_{i}} = [f, p_{i}]_{\star} = 0,
	\]
	so that $f$ is constant. 
\end{proof}
\section{DQ-algebras}
\begin{definition}[{\cite[Definition 2.2.5.]{DQM}}]\label{def:dq-algebra}
	A \textit{DQ-algebra} $\mathscr{A}$ on $X$ is a sheaf of $\C\hbarr$-algebras locally isomorphic to a star-algebra $(\mathscr{O}_{X}\hbarr, \star)$ as $\C\hbarr$-algebras. 
\end{definition}
A DQ-algebra induces a natural Poisson structure on $X$ as follows: Let $f,g \in \mathscr{O}_{X}$ and denote by $\sigma_{0}$ the composition
	\[
		\mathscr{A} \twoheadrightarrow \mathscr{A}/\hbar\mathscr{A} \xrightarrow{\sim} \mathscr{O}_{X}.
	\]
Choose $a, b \in \mathscr{A}$ such that $\sigma_{0}(a) = f$ and $\sigma_{0}(b) = g$. Since $ab - ba \in \hbar\mathscr{A}$, define	
\begin{align}\label{eqn:poisson-induced-by-dq-algebra}
		\{f,g\} := \sigma_{0}\left(\frac{ab - ba}{\hbar}\right).
\end{align}
Since $\sigma_{0}(\hbar\mathscr{A}) = 0$, it follows that \eqref{eqn:poisson-induced-by-dq-algebra} is independent of the choice of liftings. Moreover, any two locally isomorphic DQ-algebras induce the same Poisson structure. Whenever $X$ is symplectic, we say that $\mathscr{A}$ is \emph{symplectic} if it is locally isomorphic to a symplectic star-algebra.


\section{DQ-algebroids}
\subsection{Symplectic DQ-algebroids}
\begin{definition}\label{def:symplectic-DQ-algebroid} 
	
\end{definition}

%!TEX root = ../../main.tex
\chapter{Dilation equivariance}\label{ch:dilation}
Let $(X, \omega)$ be a symplectic manifold and $\mathscr{A}$ a symplectic DQ-algebra on $X$. Denote by $Z(\mathscr{A})$ the center of $\mathscr{A}$. From \cref{lemm:center-weyl-algebra} and \todo{include reference of DQ-Darboux} the natural map $\C[ [\hbar]] \to \mathscr{A}$ is injective onto the center.


\appendix
%!TEX root = ../../main.tex
\chapter{Abstract nonsense}\label{app:abstract-nonsense}
Throughout this appendix we assume familiarity with basic category theory and basic sheaf theory.
% It is important to remark that for a \textit{category} $\mathcal{C}$ and a pair of objects $x,y$ we will not require the collection of morphisms $\mathcal{C}(x,y)$ to be a set, whenever we require this condition we will say that $\mathcal{C}$ is \textit{locally small}. We denote the category of all categories\footnote{without running into paradoxes} by $\mathsf{CAT}$ and the locally small category of small categories by $\mathsf{Cat}$.
% \section{2-Categories}
% \begin{definition}\label{def:2-category}
% 	A \textit{(strict) 2-category} $\mathcal{C}$ consists of:
% 	\begin{enumerate}[label = (\roman*)]
% 		\item A collection of \textit{0-cells} or objects.
% 		\item For all objects $A$ and $B$, a category $\mathcal{C}(A,B)$. The objects $f,g\colon A \to B$ are called (1-)\textit{morphisms}, \textit{1-cells} or \textit{arrows}. Natural transformations $\alpha\colon f \Rightarrow g$ are called \textit{2-morphisms} or \textit{2-cells}.
% 		\item For any object $A$ there is a functor from the terminal category to $\mathcal{C}(A,A)$ sending the unique object to $\id{A}$ and its unique arrow to $\mathbf{1}_{\id{A}}$, the identity natural transformation.
% 		\item For all objects $A, B$ and $C$, there is a functor $\circ\colon \mathcal{C}(B,C)\times \mathcal{C}(A,B) \to \mathcal{C}(A,C)$ called \textit{horizontal composition} which is associative and admits the identity 1 and 2-cells of $\id{A}$ as identities. 
% 	\end{enumerate}
	
% \end{definition}
\section{Stacks}\label{sec:stacks}
We do not provide proofs, but reference the reader to the source \cite[Exposé VI]{SGA1}.
\begin{definition}\label{def:fibred-category}
	A \textit{fibred category} $\mathcal{C}$ over $X$ is an assignment 
	\begin{enumerate}[label = (\roman*)]
		\item for every open set $U \subseteq X$ a category $\mathcal{C}(U)$;
		\item for every inclusion $i\colon V \hookrightarrow U$ an inverse image functor $i^{*}\colon \mathcal{C}(U) \to \mathcal{C}(V)$, which may be taken to be the identity functor whenever $f = \id{U}$;
		\item a natural isomorphism $\tau_{i,j}\colon (i\circ j)^{*} \Rightarrow j^{*} \circ i^{*}$ for every pair of inclusions $W \xhookrightarrow{j} V \xhookrightarrow{i} U$.
	\end{enumerate}
	Subject to the condition that the diagram
	\begin{equation}\label{eqn:fibred-category-compatibility}
		\begin{tikzcd}
			(i\circ j \circ k)^{*} \ar[r, Rightarrow, "\tau_{i \circ j,k}"] \ar[d, Rightarrow, "\tau_{i, j\circ k}"] & k^{*}\circ(i \circ j)^{*} \ar[d, Rightarrow, "k^{*} \circ \tau_{i,j}"] \\
			(j\circ k)^{*} \circ i^{*} \ar[r, Rightarrow, "\tau_{j, k} \circ i^{*}"] & k^{*} \circ j^{*} \circ i^{*},
		\end{tikzcd}
	\end{equation}
	commutes for any three composable arrows $N \xhookrightarrow{k} W \xhookrightarrow{j} V \xhookrightarrow{i} U$. For an object $x$ of $\mathcal{C}(U)$, we denote by $x\lvert_{V}$ the inverse image $i^{*}(x)$ for an inclusion $i\colon V \hookrightarrow U$.
\end{definition}
\begin{definition}\label{def:morphisms-fibred-categories}
	Let $\mathcal{C}$ and $\mathcal{D}$ be two fibred categories over $X$. A \textit{morphism} $ (F, \alpha)\colon \mathcal{C} \to \mathcal{D}$ \textit{of fibred categories} consists of
	\begin{enumerate}[label = (\roman*)]
		\item for every open set $U \subseteq X$ a functor $ F_{U}\colon \mathcal{C}(U) \to \mathcal{D}(U)$ 
		\item for every inclusion $i\colon V \hookrightarrow V$ a natural isomorphism $\alpha_{i}\colon  F_{V} \circ i^{*} \Rightarrow i^{*} \circ  F_{U}$ 
	\end{enumerate}
	subject to the compatibility condition that the diagram
		\begin{equation}\label{eqn:morphisms-fibred-categories-compatibility}
		\begin{tikzcd}[column sep = tiny]
                                                        &  F_{W}\circ (i\circ j)^{*} \arrow[ld, Rightarrow, " F_{W}\circ \tau"] \arrow[rr, Rightarrow, "\alpha_{i\circ j}"] &                                            & (i \circ j)^{*} \circ  F_{U} \arrow[rd, Rightarrow, "\tau \circ  F_U"] &                                 \\
 F_{W}\circ j^{*} \circ i^{*} \arrow[rrd, Rightarrow, "\alpha_{j}\circ i^{*}"] &                                                                            &                                            &                                                       & j^{*}\circ i^{*} \circ  F_{U} \\
                                                        &                                                                            & j^{*}\circ F_{V}, \arrow[rru, Rightarrow, "j^{*}\circ \alpha_{i}"] &                                                       &                                
		\end{tikzcd}
		\end{equation}
		where $\tau$ denotes the corresponding natural isomorphism for $\mathcal{C}$ and $\mathcal{D}$, commutes for all inclusions $W \xhookrightarrow{j} V \xhookrightarrow{i} U$. 	
\end{definition}
\begin{definition}\label{def:weak-equivalence}
	A fibred morphism $(F, \alpha)\colon \mathcal{C} \to \mathcal{D}$ is called a \textit{weak equivalence} if every $F_{U}$ is fully faithful and \emph{locally surjective}; that is, for every $U$ open subset of $X$, $y$ of $\mathcal{D}(U)$, and $p \in X$ there exists an object $x$ of $\mathcal{C}(U)$ and an open neighborhood $V$ of $p$ contained in $U$ such that $F_{V}(x) \cong y\lvert_{V}$.
\end{definition}
\begin{definition}\label{def:fibred-transformation}
	Given two fibred morphisms $(F,\alpha),(G,\beta)\colon \mathcal{C} \to \mathcal{D}$, a \emph{fibred transformation} (or simply \textit{2-morphism}) $\Psi\colon F \Rightarrow G$ is a collection of natural transformations $\Psi_{U}\colon F_{U} \Rightarrow G_{U}$ indexed by open sets $U\subseteq X$ subject to the following compatibility condition: for any inclusion $i\colon V \hookrightarrow U$, the diagram of natural transformations 
	\begin{equation}\label{eqn:fibred-transformation-compatibility}
	\begin{tikzcd}
		F_{V} \circ i^{*} \ar[r, Rightarrow, "\alpha_{i}"] \ar[d, Rightarrow, "\Psi_{V} \circ i^{*}"] & i^{*} \circ F_{U} \ar[d, Rightarrow, "i^{*}\circ \Psi_{U}"] \\
		G_{V} \circ i^{*} \ar[r, Rightarrow, "\beta_{i}"] &  i^{*}\circ G_{U} 
	\end{tikzcd}
	\end{equation}
	commutes.
\end{definition}
\begin{remark}\label{rem:fibred-categories-form-a-2-category}
	Fibred categories over $X$ form a 2-category with objects as in \cref{def:fibred-category}, 1-morphisms as in \cref{def:morphisms-fibred-categories}, and 2-morphisms as in \cref{def:fibred-transformation}. We denote this 2-category by $\mathsf{Fibred}_{X}$.
\end{remark}
\begin{remark}\label{rem:presheaf-defined-by-fibred-category}
	Let $U$ be an open subset of $X$ and $x,y$ objects of $\mathcal{C}(U)$. The assignment $V \to \mathcal{C}(V)(x\lvert_{V}, y\lvert_{V})$ defines a presheaf on $U$. We denote this presheaf by $\uHom_{\mathcal{C}}(x,y)$. Moreover, every morphism $F\colon \mathcal{C} \to \mathcal{D}$ induces a morphism at the level of presheaves.
\end{remark}
\begin{definition}\label{def:prestack}
	A fibred category $\mathcal{C}$ over $X$ is a \textit{prestack on} $X$ if for every $U$ and every pair $x,y$ of objects of $\mathcal{C}(U)$ the presheaf $\uHom_{\mathcal{C}}(x,y)$ is a sheaf. The full 2-subcategory of $\mathsf{Fibred}_{X}$ made of prestacks is denoted by $\mathsf{Prestacks}_{X}$.
\end{definition}
\begin{remark}\label{rem:associated-prestack-of-a-fibred-category}
	Every fibred category $\mathcal{C}$ admits an \textit{associated prestack}, in the sense of a left 2-adjoint for
	\[
		\mathsf{Prestacks}_{X} \hookrightarrow \mathsf{Fibred}_{X}.
	\]
	Indeed, consider the associated sheaf (also known as sheafification) to the presheaf in \cref{rem:presheaf-defined-by-fibred-category}. The usual adjointness associated sheaf $\dashv$ presheaf extends to the desired 2-adjointness. 
\end{remark}
\begin{definition}\label{def:descend-data}
	Let $\mathcal{C}$ be a fibred category over $X$ and $\mathscr{U} = \{U_{\alpha}\}_{\alpha\in A}$ be an open cover of an open set $U\subseteq X$. The category $\mathrm{Desc}(\mathscr{U}, \mathcal{C})$ of \emph{descent data} consists of
	\begin{enumerate}[label = (\roman*)]
		\item as objects: pairs of collections $(x,\varphi) = (\{x_{\alpha}\}_{\alpha\in A}, \{\varphi_{\alpha\beta}\}_{\alpha, \beta\in A})$ where $x_{\alpha}$ is an object of $\mathcal{C}_{\alpha}$ and $\varphi_{\alpha\beta}\colon x_{\beta}\lvert_{U_{\alpha\beta}} \xrightarrow{\sim} x_{\alpha}\lvert_{U_{\alpha\beta}}$ an isomorphism. These are subject to the cocycle condition
		\begin{equation}\label{eqn:cocycle-condition}
			\varphi_{\alpha\beta} \circ \varphi_{\beta\gamma} = \varphi_{\alpha\gamma}
		\end{equation}
		in $\mathcal{C}_{U_{\alpha\beta\gamma}}$, for each $\alpha, \beta$ and $\gamma$ in $A$.
		\item as arrows: $(x, \varphi) \xrightarrow{f} (y, \psi)$ a set of arrows $\{f_{\alpha}\colon x_{\alpha} \to y_{\alpha}\}_{\alpha\in A}$ such that the diagram
		\begin{equation}\label{eqn:compatibility-descent-data}
			\begin{tikzcd}
				x_{\beta}\lvert_{U_{\alpha\beta}} \ar[r, "f_{\beta}"] \ar[d, "\varphi_{\alpha \beta}"] & y_{\beta}\lvert_{U_{\alpha\beta}} \ar[d, "\psi_{\alpha \beta}"] \\
				x_{\alpha}\lvert_{U_{\alpha\beta}} \ar[r, "f_{\alpha}"] & x_{\alpha}\lvert_{U_{\alpha\beta}} 
			\end{tikzcd}
		\end{equation}
		
	\end{enumerate}
	
\end{definition}
\begin{remark}\label{rem:fibred-category-induces-descent}
	Let $\mathcal{C}$ be a fibred category over $X$, $U$ an open set of $X$, and $\mathscr{U}$ a cover of $U$. There is a natural functor $\mathcal{C}(U) \to \mathrm{Desc}(\mathscr{U}, \mathcal{C})$ sending $x \mapsto (\{x\lvert_{U_{\alpha}}\}_{\alpha\in A}, \{\id{x}\lvert_{U_{\alpha\beta}}\}_{\alpha,\beta\in A})$ and $f\colon x \to y \mapsto \{f\lvert_{U_{\alpha}}\colon x\lvert_{U_{\alpha}} \to y\lvert_{U_{\alpha}}\}_{\alpha\in A}$. Then $\mathcal{C}$ is a prestack if and only if this functor is fully faithful for all open sets $U$ and all covers $\mathscr{U}$ of $U$.
\end{remark}
\begin{definition}\label{def:stack}
	A fibred category $\mathcal{C}$ over $X$ is a \textit{stack on} $X$ if for every open subset $U$ of $X$ and every cover $\mathscr{U}$ of $U$ the functor $\mathcal{C}(U) \to \mathrm{Desc}(\mathscr{U}, \mathcal{C})$ on \cref{rem:fibred-category-induces-descent} is an equivalence of categories. If in addition each category $\mathcal{C}(U)$ is a groupoid, we say that $\mathcal{C}$ is \textit{stack in groupoids.} We denote the full 2-subcategory of stacks on $X$ by $\mathsf{Stacks}_{X}$.
\end{definition}
\begin{definition}\label{def:associated-stack}
	Let $\mathcal{C}$ be a prestack on $X$. The \textit{associated stack} is a stack $\mathcal{C}$, endowed with a weak equivalence $F\colon\mathcal{C} \to \mathcal{C}'$ such that for every open subset $U$ of $X$ and any pair of objects $x$ and $y$ of $\mathcal{C}(U)$ the map
		\[
			\mathcal{C}(U)(x,y) \to \mathcal{C}'(U)(F_{U}(x), F_{U}(y))
		\]
	is a bijection. If $\mathcal{C}'$ exists it is determined up to unique 2-isomorphism. 
\end{definition}
\begin{proposition}\label{prop:stackification}
	Let $\mathcal{C}$ be a prestack on $X$, then $\mathcal{C}$ admits an associated stack. 
\end{proposition}
\begin{proof}
	Let $\mathcal{C}'(U) := \colim_{\mathscr{U}}\mathrm{Desc}(\mathscr{U}, \mathcal{C})$ understood as a pseudo-colimit of categories. For details see \cite[\href{https://stacks.math.columbia.edu/tag/02ZN}{Tag 02ZN}]{stacks-project}. 
\end{proof}
\begin{remark}\label{rem:stackification-is-faithful-and-locally-full}
	From the definition of associated stack, the associated stack of a stack is equivalent to itself. Therefore, from general nonsense, we conclude that stackification is faithful. Moreover, since $\mathcal{C} \to \mathcal{C'}$ is fully faithful and locally surjective, then stackification is locally full. 
\end{remark}
\todo{Write the correct adjunction or 2-adjunction}
\section{Algebroid stacks}
This section tries to follow \cite[Section 1]{dagnolo-polesello_complex-involutive-submanifolds}. Let $\mathbb{K}$ be a commutative unital ring. 
\begin{definition}\label{def:linear-category}
	A $\mathbb{K}$-\emph{linear category} is a category whose $\operatorname{Hom}$ sets are endowed with a $\mathbb{K}$-module structure, so that composition is bilinear. A $\mathbb{K}$-\emph{functor} is a functor between $\mathbb{K}$-categories which is linear at the level of morphisms. \todo{write in terms of enriched categories might kill two birds with one shot... linear categories and 2-categories}
\end{definition}
\begin{remark}\label{rem:linear-yoneda-lemma}
	Any $\mathbb{K}$-linear category admits a $\mathbb{K}$-Yoneda embedding, that is, one embeds it on its category of $\mathsf{Mod}(\mathbb{K})$-valued $\mathbb{K}$-functors. 
\end{remark}
\begin{example}\label{ex:linear-category-of-modules}
	Let $A$ be a $\mathbb{K}$-algebra, then the category of left $A$-modules $\mathsf{Mod}(A)$ is a $\mathbb{K}$-category.
\end{example}
\begin{example}\label{ex:linear-category-associated-to-an-alegbra}
	Let $A$ be a $\mathbb{K}$-algebra, we denote by $A^{+}$ the category of one object $\bullet$ such that $A^{+}(\bullet,\bullet) = A$. Given $B$ another $\mathbb{K}$-algebra, any linear map $f\colon A \to B$ induces a $\mathbb{K}$-functor $f^{+}\colon A^{+} \to B^{+}$. This defines a fully faithful functor between $\mathbb{K}$-algebras and (small) $\mathbb{K}$-categories. Moreover, any transformation $f^{+} \Rightarrow g^{+}$ correspond to elements $b\in B$ such that $bf(a) = g(a)b$ for all $a \in A$.
\end{example}
These two examples are related by the following proposition.
\begin{proposition}\label{prop:yoneda-for-linear-categories}
	Let $A$ be a $\mathbb{K}$-algebra. The category of left $A$-modules $\mathsf{Mod}(A)$ is equivalent to the category of $\mathbb{K}$-functors $\mathsf{Hom}_{\mathbb{K}}(A^{+}, \mathsf{Mod}(\mathbb{K}))$ from $A^{+}$ to the category of left $\mathbb{K}$-modules $\mathsf{Mod}(\mathbb{K})$. Moreover, via the $\mathbb{K}$-Yoneda embedding
	\[
		A^{+} \to \mathsf{Hom}_{\mathbb{K}}( (A^{+})^{\op}, \mathsf{Mod}(\mathbb{K}) ) \approx_{\mathbb{K}} \mathsf{Mod}(A^{\op})
	\]
	$A^{+}$ is identified with the full subcategory of right $A$-modules which are free of rank one.
\end{proposition}
\begin{proof}
	Given an $A$-module $M$ we may consider the $\mathbb{K}$-functor $A^{+} \to \mathsf{Mod}(\mathbb{K})$ which sends $\bullet \mapsto M_{\mathbb{K}}$, that is $M$ as a $\mathbb{K}$-module, and at the level of morphisms $A \to \mathrm{End}_{\mathbb{K}}(M)$ maps $a$ to multiplication by $a$. On the other hand, given a $\mathbb{K}$-functor $F\colon A^{+} \to \mathsf{Mod}(\mathbb{K})$, the $\mathbb{K}$-module $F(\bullet)$ admits an $A$-structure via the map $A \to \mathrm{End}_{\mathbb{K}}(M)$ given by $F$ at the level of morphisms. These two processes are strictly invertible, and \textit{a fortiori} form an equivalence. The second part is obvious if one notices $(A^{+})^{\op} = (A^{\op})^{+}.$
\end{proof}

Let $X$ be a topological space. 
\begin{definition}\label{def:linear-stack}
	A $\mathbb{K}$-\textit{fibred category} over $X$ is a fibred category $\mathcal{C}$ such that
	\begin{enumerate}[label = (\roman*)]
		\item $\mathcal{C}(U)$ is a $\mathbb{K}$-linear category for all $U$;
		\item $i^{*}\colon \mathcal{C}(U) \to \mathcal{C}(V)$ is a $\mathbb{K}$-functor for all inclusions $V \xhookrightarrow{i} U$.
	\end{enumerate}
	Analogously, we extend it to $\mathbb{K}$-\textit{(pre)stacks.} A morphism $(F,\alpha)\colon \mathcal{C} \to \mathcal{D}$ between $\mathbb{K}$-fibred categories is a $\mathbb{K}$-\textit{morphism} if $F$ induces $\mathbb{K}$-functors $F_{U}\colon \mathcal{C}(U) \to \mathcal{D}_{U}$ for all $U$.
\end{definition}
\begin{example}\label{ex:linear-stack-associated-to-an-algebra}
	Let $\mathscr{A}$ be a sheaf of $\mathbb{K}$-algebras. The assignment $U \mapsto \mathscr{A}(U)^{+}$ defines a $\mathbb{K}$-prestack. Indeed, if $\mathscr{A}$ is just a presheaf, then $\mathscr{A}^{+}$ defines a $\mathbb{K}$-fibred category and the 2-morphisms $\tau_{i,j}$ in \cref{def:fibred-category} may be chosen to be the identity natural transformation in a such a way that \eqref{eqn:fibred-category-compatibility} translates that $\mathscr{A}$ defines functor. Moreover, the presheaf on \cref{rem:presheaf-defined-by-fibred-category} for $\mathscr{A}(U)^{+}$ is a sheaf for all $U$ if and only if $\mathscr{A}$ is a sheaf. In addition, a map $f\colon \mathscr{A} \to \mathscr{B}$ of sheaves induces $\mathbb{K}$-morphisms of the corresponding prestacks. In this case, we take $\alpha_{i}$ in \cref{def:morphisms-fibred-categories} to be the identity natural transformation so that 
	In order to achieve a compact data, we may assume that\eqref{eqn:morphisms-fibred-categories-compatibility} translates that $f$ is a morphism of presheaves. The associated stack to $U \mapsto \mathscr{A}(U)^{+}$ is denoted by $\mathscr{A}^{+}$. From \cref{rem:stackification-is-faithful-and-locally-full} any $\mathbb{K}$-functor $\Phi\colon\mathscr{A}^{+} \to \mathscr{B}^{+}$ is locally induced by a map of $\mathbb{K}$-algebras. More precisely, there exists a cover $\mathscr{U} = \{U_{\alpha}\}_{\alpha}$ of $X$ such that $\Phi\lvert_{U_{\alpha}} = (f_{\alpha})^
	In order to achieve a compact data, we may assume that{+}$ for some morphism $f_{\alpha}\colon\mathscr{A}\lvert_{U_{\alpha}} \to \mathscr{B}\lvert_{U_{\alpha}}$ of sheaves of $\mathbb{K}$-algebras. 
\end{example}
\begin{example}\label{ex:modules-over-sheaf}
	Let $\mathscr{A}$ be a sheaf of $\mathbb{K}$-algebras. The assignment $U \mapsto \mathsf{Mod}(\mathscr{A}\lvert_{U})$ defines a $\mathbb{K}$-stack on $X$ which we denote by $\mathfrak{Mod}(\mathscr{A})$.
\end{example}
These two examples are related by the analogue of \cref{prop:yoneda-for-linear-categories}.
\begin{proposition}\label{prop:yoneda-for-stacks}
	Let $\mathscr{A}$ be a sheaf of $\mathbb{K}$-algebras. The stack of left $\mathscr{A}$-modules $\mathfrak{Mod}(\mathscr{A})$ is equivalent to the stack of $\mathbb{K}$-functors $\mathfrak{Hom}_{\mathbb{K}}(\mathscr{A}^{+}, \mathfrak{Mod}(\mathbb{K}_{X}))$ from $\mathscr{A}^{+}$ to the category of left $\mathbb{K}_{X}$-modules $\mathfrak{Mod}(\mathbb{K}_{X})$. Moreover, via the $\mathbb{K}$-Yoneda embedding of stacks
	\[
		\mathscr{A}^{+} \to \mathfrak{Hom}_{\mathbb{K}}( (\mathscr{A}^{+})^{\op}, \mathfrak{Mod}(\mathbb{K}_{X}) ) \approx_{\mathbb{K}} \mathfrak{Mod}(\mathscr{A}^{\op})
	\]
	$\mathscr{A}^{+}$ is identified with the full substack of right $\mathscr{A}$-modules which are locally free of rank one. \todo{Need to state correctly what does ``stack of $\mathbb{K}$-functors'' for large objects mean}
\end{proposition}
\begin{definition}\label{def:non-empty}
	Let $\mathcal{C}$ be fibred category over $X$. We say that $\mathcal{C}$ is \textit{non-empty} if $\mathcal{C}(X)$ has at least one object. It is \textit{locally non-empty} if there exists an open covering $\mathscr{U} = \{U_{\alpha}\}_{\alpha\in A}$ of $X$ such that $\mathcal{C}\lvert_{U_{\alpha}}$ is non-empty. 
\end{definition}
\begin{definition}\label{def:locally-connected}
	Let $\mathcal{C}$ be a fibred category. We say that $\mathcal{C}$ is \textit{locally connected} if for every open subset $U \subseteq X$ and any pair of objects $x, y$ of $\mathcal{C}(U)$ there exists an open cover $\mathscr{U} = \{U_{\alpha}\}_{\alpha\in A}$ of $U$ such that $x\lvert_{U_{\alpha}} \simeq y\lvert_{U_{\alpha}}$ in $\mathcal{C}(U_{\alpha})$.
\end{definition}
\begin{definition}\label{def:algebroid-stack}
	A $\mathbb{K}$-\emph{algebroid stack} is a $\mathbb{K}$-stack which is locally non-empty and locally connected.
\end{definition}
\begin{example}\label{ex:algebroid-stack-defined-by-sheaf}
	Let $\mathscr{A}$ be a sheaf of $\mathbb{K}$-algebras. The stack $\mathscr{A}^{+}$ is an algebroid stack. Indeed, from \cref{prop:yoneda-for-stacks} $\mathscr{A}^{+}$ is $\mathbb{K}$-equivalent to the stack of locally free right $\mathscr{A}$-modules, which is evidently non-empty and locally connected.
\end{example}
The following proposition shows that one cannot strengthen local non-emptiness in \cref{def:algebroid-stack} to global non-emptiness and expect a non-trivial answer, in the sense of \cref{ex:algebroid-stack-defined-by-sheaf}.
\begin{proposition}[{\cite[Lemma 1.1.]{dagnolo-polesello_complex-involutive-submanifolds}}]\label{prop:non-empty-algebroid-stack-is-trivial}
	Let $\mathcal{C}$ be a non-empty and locally connected $\mathbb{K}$-stack. For $x$ an object of $\mathcal{C}(X)$, $\mathcal{C}$ is $\mathbb{K}$-equivalent to $\uEnd_{\mathcal{C}}(x)^{+}$.
\end{proposition}
\begin{proof}
	Let $\mathscr{A}$ denote the sheaf of $\mathbb{K}$-algebras $\uEnd_{\mathcal{C}}(x)$. For $U \subseteq X$ an open subset, consider the assignment $y \mapsto \uHom_{\mathcal{C}\lvert_{U}}(x\lvert_{U}, y)$ between $\mathcal{C}(U)$ and right $\mathscr{A}\lvert_{U}$-modules, which are locally free of rank one, defines a $\mathbb{K}$-equivalence $\mathcal{C} \to \mathscr{A}^{+}$ via \cref{prop:yoneda-for-stacks}. Indeed, for the quasi-inverse, send the unique object of $\mathscr{A}(U)^{+}$ to $x\lvert_{U}$ and at the level of morphisms $\mathscr{A}(U) \to \mathcal{C}(U)(x\lvert_{U}, x\lvert_{U})$ is the identity. This induces a map between the associated stack $\mathscr{A}^{+} \to \mathcal{C}$ which is the desired quasi-inverse.
\end{proof}
Now, we describe algebroid stacks via local data. Let $\mathcal{C}$ be a $\mathbb{K}$-algebroid stack. By definition, there exists a covering $\mathscr{U} = \{U_{\alpha}\}_{\alpha\in A}$ such that $\mathcal{C}\lvert_{U_{\alpha}}$ is non-empty and locally connected. Let $x_{\alpha}$ be an object of $\mathcal{C}(U_{\alpha})$ and $\mathscr{A}_{\alpha} := \uEnd_{\mathcal{C}\lvert_{U_{\alpha}}}(x_{\alpha})$, from \cref{prop:non-empty-algebroid-stack-is-trivial} we obtain a $\mathbb{K}$-equivalence $\Phi_{\alpha}\colon \mathcal{C}\lvert_{U_{\alpha}} \to \mathscr{A}_{\alpha}^{+}$ with quasi-inverse $\Psi_{\alpha}\colon \mathscr{A}_{\alpha}^{+} \to \mathcal{C}\lvert_{U_{\alpha}}$. On double intersections $U_{\alpha\beta}$ denote by $\Phi_{\alpha\beta}$ the composition $\Phi_{\alpha}\circ\Psi_{\beta}\colon \mathscr{A}_{\beta}^{+}\lvert_{U_{\alpha\beta}} \to \mathscr{A}_{\alpha}^{+}\lvert_{U_{\alpha \beta}}$. On triple intersections $U_{\alpha\beta\gamma}$ the natural isomorphism $\Psi_{\beta}\circ\Phi_{\beta} \Rightarrow \id{\mathcal{C}\lvert_{U_{\alpha\beta}}}$ induces 2-isomorphisms $\theta_{\alpha\beta\gamma}\colon \Phi_{\alpha\beta} \circ \Phi_{\beta\gamma} \Rightarrow \Phi_{\alpha\gamma}$ such that on quadruple intersections $U_{\alpha\beta\gamma\delta}$ the diagram
\begin{equation}\label{eqn:local-compatibility-algebroid}
	\begin{tikzcd}
\Phi_{\alpha\beta}\circ\Phi_{\beta\gamma} \circ \Phi_{\gamma\delta} \arrow[rr, "\theta_{\alpha\beta\gamma} \circ \id{\Phi_{\gamma\delta}}", Rightarrow] \arrow[d, "\id{\Phi_{\alpha\beta}} \circ \theta_{\beta\gamma\delta}", Rightarrow] &  & \Phi_{\alpha\gamma}\circ \Phi_{\gamma\delta} \arrow[d, "\theta_{\alpha\gamma\delta}", Rightarrow] \\
\Phi_{\alpha\beta}\circ \Phi_{\beta\delta} \arrow[rr, "\theta_{\alpha\beta\delta}", Rightarrow]                                                                              &  & \Phi_{\alpha\delta}                                                                              
\end{tikzcd}
\end{equation}
commutes. From \cref{rem:stackification-is-faithful-and-locally-full} there exists a cover $\mathscr{U}_{\alpha\beta} = \{U_{\alpha\beta}^{i}\}_{i\in I}$ such that $\Phi_{\alpha\beta}\lvert_{U_{\alpha\beta}^{i}} = (f_{\alpha\beta}^{i})^{+}$ for some isomorphisms of $\mathbb{K}$-algebras $f_{\alpha\beta}^{i}\colon \mathscr{A}_{\beta}\lvert_{U_{\alpha\beta}^{i}} \to \mathscr{A}_{\alpha}\lvert_{U_{\alpha\beta}^{i}}$. On triple intersections $U_{\alpha\beta\gamma}^{ijk} = U_{\alpha\beta}^{i}\cap U_{\alpha\gamma}^{j}\cap U_{\beta\gamma}^{k}$ we have 2-isomorphisms $\theta_{\alpha\beta\gamma}\lvert_{U_{\alpha\beta\gamma}^{ijk}}\colon (f_{\alpha\beta}^{i})^{+}\circ (f_{\beta\gamma}^{k})^{+} \Rightarrow (f_{\alpha\gamma}^{j})^{+}$. Therefore, there are sections $a_{\alpha\beta\gamma}^{ijk}\in \mathscr{A_{\alpha}}^{\times}(U_{\alpha\beta\gamma}^{ijk})$ such that 
\[
	f_{\alpha\beta}^{i} \circ f_{\beta\gamma}^{k} = \ad(a^{ijk}_{\alpha \beta \gamma})f_{\alpha\gamma}^{j}.
\]
On quadruple intersections $U_{\alpha\beta\gamma \delta}^{ijklmn} = U_{\alpha \beta \gamma}^{ijk} \cap U_{\alpha \beta \delta}^{ilm} \cap U_{\alpha \gamma \delta}^{jln}\cap U_{\beta \gamma \delta}^{kmn}$ the diagram \eqref{eqn:local-compatibility-algebroid} translates to the equation
\[
	a_{\alpha \beta \gamma}^{ijk}a_{\alpha \beta \gamma}^{jln} = f_{\alpha \beta}^{i}(a_{\beta \gamma \delta}^{kmn})a_{\alpha \beta \delta}^{iln}
\]
In order to achieve a compact local description, we may assume that $X$ is paracompact and use fact that hypercoverings are cofinal among coverings \cite[Chapitre II, Lemme 3.8.1.]{Godement}. Therefore, we get the following local characterization
\begin{proposition}[{\cite[Proposition 2.1.]{dagnolo-polesello_complex-involutive-submanifolds}}, {\cite[Proposition 2.1.3.]{DQM}}]\label{prop:local-data-algebroid}
	Assume $X$ is paracompact. For gluing datum consisting of
	\begin{enumerate}[noitemsep, label = (\roman*)]
		\item an open cover $\mathscr{U} = \{U_{\alpha}\}_{\alpha\in A}$;
		\item $\mathbb{K}$-algebras $\mathscr{A}_{\alpha}$ on $U_{\alpha}$;
		\item isomorphisms of $\mathbb{K}$-algebras $f_{\alpha\beta}\colon \mathscr{A}_{\beta} \to \mathscr{A}_{\alpha}$ on $U_{\alpha\beta}$ and;
		\item sections $a_{\alpha \beta \gamma}\in \mathscr{A}^{\times}(U_{\alpha \beta \gamma})$
	\end{enumerate}
	subject to the conditions on morphisms
	\begin{equation}\label{eqn:gluing-datum-algebroid-morphisms}
		f_{\alpha\beta} \circ f_{\beta\gamma} = \ad(a_{\alpha \beta \gamma})f_{\alpha\gamma}
	\end{equation}
	and on sections
	\begin{equation}\label{eqn:gluing-datum-algebroid-sections}
		a_{\alpha \beta \gamma}a_{\alpha \beta \gamma} = f_{\alpha \beta}(a_{\beta \gamma \delta})a_{\alpha \beta \delta}
	\end{equation}
	there exists a $\mathbb{K}$-algebroid on $X$ to which this gluing datum is associated. Moreover, the data is unique up to equivalence of stacks, with this equivalence unique up to unique isomorphism.
	
\end{proposition}



% ====================================================
%
% ENDMATTER
%
% Appendices and bibliography 
% Pagination arabic, re-starts at 1
%
% ====================================================
\cleardoublepage % start afresh on a new page
%\appendixpage* % makes a page to mark beginning of appendices
% \input{appendix1} 


\printbibliography[title={References}, heading=bibintoc]


\end{document}