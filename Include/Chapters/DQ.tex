%!TEX root = ../../main.tex
\chapter{dq-algebras and dq-algebroids}\label{ch:DQ}
\section{Star products}
\todo{some words on history...}
\begin{definition}[{\cite{BFFLS}, \cite[Definition~2.2.2.]{DQM}}]\label{def:star-product}
	An associative operation $\star$ on $\mathscr{O}_{X}[[\hbar]]$ is a \textit{star product} if it is $\C[[\hbar]]$-bilinear and satisfies
	\begin{align}\label{eqn:star-product}
		f \star g = \sum_{i\geq 0} P_{i}(f,g)\hbar^{i} \quad \text{for } f,g\in \mathscr{O}_{X}
	\end{align}
	where the $P_{i}$'s are bi-differential operators such that $P_{0}(f, g)=fg$ and $P_{i}(f, 1)=P_{i}(1, f)=0$ for all $f \in \mathscr{O}_{X}$ and $i>0 .$ We call $\left(\mathscr{O}_{X}[[\hbar]], \star\right)$ a \textit{star algebra}. A star-product defines a Poisson structure on $X$ by the rule
	\begin{align}\label{eqn:poisson-induced-by-star-product}
		\{f,g\} := P_{1}(f,g) - P_{1}(f,g) = \frac{1}{\hbar}[f,g]_{\star} \mod \hbar\mathscr{O}_{X} 
	\end{align}
	If $X$ is a Poisson manifold and the induced Poisson bracket by $\star$ coincides with the Poisson structure of $X$, we say that $\star$ is a \textit{deformation} of the Poisson structure (bracket) on $X$. Moreover, if $(X,\omega)$ is symplectic and $\star$ deforms the standard Poisson bracket induced by $\omega$, we say that $\star$ is a \textit{symplectic star product}.
\end{definition}
\todo{the previous probably needs some background on differential operators...}
\subsection{Moyal product}
\begin{definition}\label{def:moyal}
	Let $X = \mathbb{R}^{2n}$ endowed with its standard symplectic structure. We define the \textit{Moyal product} by the rule
    \begin{align}\label{eqn:weyl-moyal-free}
    	f \star g &= \operatorname{prod}\left(\exp \left(\frac{\hbar}{2}\Pi\right)(f \otimes g)\right) \\
    	&= fg + \frac{\hbar}{2}\sum_{i,j}\Pi^{ij}(\partial_{i}f)(\partial_{j}g) + \frac{\hbar^{2}}{8} \sum_{i,j,k,m} \Pi^{ij}\Pi^{km}(\partial_{i}\partial_{k}g)(\partial_{j}\partial_{m}g) + \ldots, \nonumber
    \end{align}
	where $\Pi$ is the Poisson bi-vector. One calls $\mathscr{O}_X[[\hbar]]$ equipped with the Moyal product, the \textit{Weyl algebra}.
\end{definition}
\begin{remark}\label{rem:weyl-algebra}
	The notation Weyl algebra in \cref{def:moyal} is supported in the following observation. Consider the subalgebra $(\C[p,q], \star)$\footnote{Here $p$ and $q$ are shorthand for coordinates $(p_{1}, \ldots, p_{n}, q_{1}, \ldots, q_{n})$} of the algebra $(\mathscr{O}_{\R^{2n}}(\R^{2n}), \star)$. An easy computation yields 
	\[
		p_{r} \star q_{s} = p_{s}q_{s} + \delta_{rs}\frac{\hbar}{2} \quad \text{and,} \quad q_{s} \star p_{r} = p_{r}q_{s} - \delta_{sr}\frac{\hbar}{2}.
	\]
	Therefore,
	\[
		[p_{r},q_{s}]_{\star} = \delta_{rs}\hbar.
	\]
	Particularly, this presents an isomorphism between $(\C[p,q],\star)$ and\footnote{We mantain the same shorthand as before, and we extend it correctly to $x\partial - \partial x - 1$} $\C\{x,\partial\}/\left\langle x\partial - \partial x - 1\right\rangle$, also known as the Weyl algebra. This will be further explained by the identification of $\R^{2n}$ with $T^{*}\R^{n}$ and deformation quantization of the cotangent bundle.
\end{remark}
\begin{lemma}\label{lemm:center-weyl-algebra}
	The center of the algebra $(\mathscr{O}_{\R^{2n}}, \star)$ is $\C[[\hbar]]$. 
\end{lemma}
\begin{proof}
	Choose coordinates $(p_{i},q_{i})$ and let $f \in (\mathscr{O}_{\R^{2n}}, \star)$ be central. Then
	\[
		0 = [f,q_{i}]_{\star} = -\hbar\frac{\partial f}{\partial q_{i}}\quad \text{and,}\quad \hbar\frac{\partial f}{\partial q_{i}} = [f, p_{i}]_{\star} = 0,
	\]
	so that $f$ is constant. 
\end{proof}
\section{DQ-algebras}
\begin{definition}[{\cite[Definition 2.2.5.]{DQM}}]\label{def:dq-algebra}
	A \textit{DQ-algebra} $\mathscr{A}$ on $X$ is a sheaf of $\C\hbarr$-algebras locally isomorphic to a star-algebra $(\mathscr{O}_{X}\hbarr, \star)$ as $\C\hbarr$-algebras. 
\end{definition}
A DQ-algebra induces a natural Poisson structure on $X$ as follows: Let $f,g \in \mathscr{O}_{X}$ and denote by $\sigma_{0}$ the composition
	\[
		\mathscr{A} \twoheadrightarrow \mathscr{A}/\hbar\mathscr{A} \xrightarrow{\sim} \mathscr{O}_{X}.
	\]
Choose $a, b \in \mathscr{A}$ such that $\sigma_{0}(a) = f$ and $\sigma_{0}(b) = g$. Since $ab - ba \in \hbar\mathscr{A}$, define	
\begin{align}\label{eqn:poisson-induced-by-dq-algebra}
		\{f,g\} := \sigma_{0}\left(\frac{ab - ba}{\hbar}\right).
\end{align}
Since $\sigma_{0}(\hbar\mathscr{A}) = 0$, it follows that \eqref{eqn:poisson-induced-by-dq-algebra} is independent of the choice of liftings. Moreover, any two locally isomorphic DQ-algebras induce the same Poisson structure. Whenever $X$ is symplectic, we say that $\mathscr{A}$ is \emph{symplectic} if it is locally isomorphic to a symplectic star-algebra.


\section{DQ-algebroids}
\subsection{Symplectic DQ-algebroids}
\begin{definition}\label{def:symplectic-DQ-algebroid} 
	
\end{definition}
